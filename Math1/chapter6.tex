\documentclass{article}
\usepackage[utf8]{inputenc}
\usepackage{textcomp}
\usepackage{gensymb}
\usepackage{graphicx}
\usepackage{multicol}
\usepackage{amsmath}
\usepackage{amssymb}
\usepackage{physics}
\usepackage{hyperref}
\usepackage{cancel}
\usepackage{scalerel}

\def\msquare{\mathord{\scalerel*{\Box}{gX}}}

\graphicspath{ {./chap6images/} }

\oddsidemargin=-.3in
\evensidemargin=-.5in
\textwidth=7in
\topmargin=-1in
\textheight=10in

\parindent=.2in
\pagestyle{plain}

\title{\textbf{The Exponent} \\ \large Math 1 - Chapter 6}
\author{Theo Rode}
\date{}

\begin{document}

\maketitle 

Say you had a bunch of people and 1 zombie. Every day, the number of zombies doubles. 
How many zombies will there be on day 5? What about day 10? How are the number of zombies growing? Does it look like addition? What about multiplication? 

Well, let's look at each day. On day 1, well we just start with 1 zombie, so we only have 1 zombie. But then, on day 2, the number of zombies doubles. This means we have $1 \cdot 2$ 
zombies... So 2 zombies! Then on the next day, the number of zombies doubles again, this gives us $2 \cdot 2$ zombies... 4 zombies! We can look at each day: 
\[ \text{day 1} \to 1 \]
\[ \text{day 2} \to 2 \]
\[ \text{day 3} \to 4 \]
\[ \text{day 4} \to 8 \]
\[ \text{day 5} \to 16 \]
\[ \text{day 6} \to 32 \]
\[ \text{day 7} \to 64 \]
\[ \ldots \]

Each time we increase the day, we multiply by 2 again. Which is how we defined it, so that isn't entirely unexpected. 
If we write this out as an equation up to, say, day 7, it would look like: 
\[ 1 \cdot 2 \cdot 2 \cdot 2 \cdot 2 \cdot 2 \cdot 2 = 64 \]

We can get rid of the 1 because it is the multiplicative identity and isn't doing anything, and that leaves us with the equation: 
\[ 2 \cdot 2 \cdot 2 \cdot 2 \cdot 2 \cdot 2 = 64 \]

Doesn't this look familiar? It is just repeated multiplication. And if you remember, we defined multiplication as repeated addition. For example: 
\[ 5 \cdot 7 = 5 + 5 + 5 + 5 + 5 + 5 + 5 \]

And this looks very similar, just with multiplication this time instead of addition. This is why we give this a special name, and it is actually the third operation we will learn together: 
\begin{center}
    \textbf{Exponentiation}
\end{center}

We notate exponents with a little number above and to the right of a number. For example: 
\[ 2^6 \]

Here, the 2 is the number which is being multiplied many times and 6 is the number of times we are multiplying it. So, in this example we have: 
\[ 2^6 = 2 \cdot 2 \cdot 2 \cdot 2 \cdot 2 \cdot 2 = 64 \]

We can put any number in either spot: 
\[ 10^4 = 10 \cdot 10 \cdot 10 \cdot 10 = 10000 \]
\[ 7^2 = 7 \cdot 7 \]
\[ 3^5 = 3 \cdot 3 \cdot 3 \cdot 3 \cdot 3 = 243 \]

This is a very useful operation for many types of models and situations (like the super relatable example of a zombie invasion)
and we will explore it for a bit today. 

For starters, something that is really important to understand is that for previous operations, we could do this: 
\[ 4 + 5 = 5 + 4 \]
\[ 3 \cdot 2 = 2 \cdot 3 \]

And this was true for any number: 
\[ a + b = b + a \]
\[ a \cdot b = b \cdot a\]

But this isn't the case with exponents. It is \textbf{not true} that $a^b = b^a$. We can pretty easily show this isn't true with 3 and 2 as we have: 
\[ 3^2 = 3 \cdot 3 = 9 \]
\[ 2^3 = 2 \cdot 2 \cdot 2 = 8 \]

8 and 9 are not the same number! This means we have to be more careful about algebra with exponents. 

We want to think about an exponent, $a^b$, as an operation where $b$ acts on $a$. So then the question is, much like how we asked this for multiplication and addition, what is the identity and inverse? 

For the identity, this means we are looking for some value of $b$ such that: 
\[ a^b = a \]

Or in other words, taking the exponent of $b$ on $a$ still returns $a$ no matter what $a$ is. 
How can we think about finding this? Well we just want $a$, and exponents represent $a$ multiplied $b$ times. But if we don't multiply $a$ by anything, ie only multiply $a$ once, we would get: 
\[ a^1 = a \] 

Cool! 1 is our identity! 

\section*{Problems}
\begin{multicols*}{2}
\begin{enumerate}
    \item What is $2^4 \cdot 2^2$? As a number? What about in the form: $2^b$?
    \item What is $3^2 \cdot 3^3$ in the form of $3^b$? 
    \item What is the value of $\left(2^3\right)^2$? As a number? What about in the form $2^b$? 
    \item What is the value of $\left(3^2\right)^3$? As a number? What about in the form $3^b$? 
    \item What is the value of $\left(3^2\right)^{\frac{1}{2}}$? As a number? What about in the form $3^b$? 
    \item What is the value of $\left(3^3\right)^{\frac{1}{3}}?$ As a number? What about in the form $3^b$? 
    \item What is the value of $2^4 \cdot 2^{-4}$? As a number? In the form $2^b$? 
    \item What is $2^0$? (Hint: Look back to the very first way we defined exponents.)
    \item What is the value of $2^3 \cdot 2^{-2}$? As a number? In the form $2^b$? 
    \item Say we had some inverse operation of exponents such that for some number, $a^b$, we could get back $a$. And say we had another function that would give us $b$. As these the same function? Are they similar? Why or why not? Can you tell me what they are? 
    \item For some number $a^b$, can we apply some exponent to get $a$. In other words, find the inverse of exponentiating by $b$? 
    \item For some number $a^b$, can we apply some exponent to get $b$? In other words, find the inverse of having the base of $a$ in an exponent?
    \item What is $2^2 \cdot 4^3$? As a number? What about in the form $2^b$? What about $4^b$? 
    \item What is $3^2 \cdot 9^3$? As a number? What about in the form $3^b$? What about $9^b$?  
\end{enumerate}
\end{multicols*}

\end{document}